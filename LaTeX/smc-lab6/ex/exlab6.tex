\documentclass[12pt]{article}

\usepackage[brazilian]{babel}
\usepackage[utf8x]{inputenc}
\usepackage[T1]{fontenc}
\usepackage{graphicx}
\usepackage{url}
\usepackage[a4paper, top=2cm, bottom=2cm, left=1.5cm, right=1.5cm]{geometry}
\usepackage{color}
\usepackage{hyperref}
\usepackage{fancyvrb}

%%%%%%%%%%%%%%%%%%%%%%%%%%%%%% DOCUMENT VARIABLES %%%%%%%%%%%%%%%%%%%%%%%%%%%%%%%%%%%%%%%
\def\doctitle{Comunicação UDP/IP}
\def\labnumber{6}
\def\fulltitle{Laboratório \labnumber -- \doctitle}
\def\docauthor{Thiago José Michelin}
\def\docsubj{Sistemas Microcomputadorizados}
\def\docsubjmn{SMC}

%%%%%%%%%%%%%%%%%%%%%%%%%%%%%% CONFIGURATION FOR GRAPHICX PACKAGE %%%%%%%%%%%%%%%%%%%%%%%
\graphicspath{{../img}}

%%%%%%%%%%%%%%%%%%%%%%%%%%%%%% COLOR CONFIGURATION FOR THE TEXT %%%%%%%%%%%%%%%%%%%%%%%%%
\definecolor{org}{RGB}{255, 102, 0}     % Orange
\definecolor{gre}{RGB}{0, 128, 0}       % Green
\definecolor{rdy}{RGB}{204, 0, 0}       % Red
\definecolor{blu}{rgb}{0, 0.33, 0.83}   % Blue
\definecolor{pur}{RGB}{153, 0, 153}     % Purple

%%%%%%%%%%%%%%%%%%%%%%%%%%%%%% CONFIGURATION FOR HYPERREF PACKAGE %%%%%%%%%%%%%%%%%%%%%%%
\hypersetup{
  colorlinks=true,
  breaklinks=true,
  linkcolor=red,
  citecolor=red,
  urlcolor=blu,
  pdftitle={\fulltitle},
  pdfauthor={\docauthor}
}

%%%%%%%%%%%%%%%%%%%%%%%%%%%%%%%%%%%%%%%%%%%%%%%%%%%%%%%%%%%%%%%%%%%%%%%%%%%%%%%%%%%%%%%%%

\newcommand{\ebt}[2]{\textcolor{#1}{\texttt{#2}}}
\newcommand{\ebf}[2]{\textcolor{#1}{\textbf{#2}}}
\newcommand{\ec}[2]{\textcolor{#1}{#2}}

\title{Laboratório \labnumber \\ \doctitle \\ \vspace{0.6cm} Exercícios}
\author{\docsubj \\
  {\small Universidade Estadual Paulista ``Júlio de Mesquita Filho''} \\
  {\small Faculdade de Engenharia e Ciências de Guaratinguetá} \\
  {\small Departamento de Engenharia Elétrica}}
\date{}

\begin{document}
\maketitle

%%%%%%%%%%%%%%%%%%%%%%%%%%%%%%%%%%%%%%%%%%%%%%%%%%%%%%%%%%%%%%%%%%%%%%%%%%%%%%%%%%%%%%%%%

Para os exercícios a seguir:

\begin{itemize}
  \item Elabore fluxogramas que representem os algoritmos que serão implementados;
  \item \textbf{Consulte a documentação} das bibliotecas utilizadas especificamente
    para implementar a comunicação UDP/IP;
  \item Ao final, execute os programas desenvolvidos e avalie se o funcionamento do
    sistema está de acordo com o que foi solicitado no enunciado do exercício.
\end{itemize}

\vspace{12pt}

\begin{itemize}
  \item[1)] Elabore um programa cliente, que utilize o protocolo \texttt{UDP},
    e seja capaz de se comunicar com um servidor \texttt{UDP} remoto com
    endereço \texttt{68.183.28.251}. Este servidor executa uma instância do
    processo conhecido como \textit{echoserver}, na porta 6666. O aplicativo
    cliente deve ser capaz de:
    
    \begin{itemize}
      \item[1.] Enviar mensagens (com qualquer conteúdo) a cada 400 ms para o servidor
        pelo período de 20 s.
      \item[3.] Medir o intervalo de tempo ($ \Delta_t $) entre o momento em que
        o cliente faz a requisição ($ t_{req} $) e o momento em que recebe uma
        resposta ($ t_{res} $) do servidor ($ \Delta_t = t_{res} - t_{req} $).
      \item[4.] Salvar, em um arquivo texto simples (\texttt{.dat}), os valores de
        $ \Delta_t $ e $ t $ obtidos no item anterior.
    \end{itemize}

  \item[2)] De forma análoga ao exercício anterior, elabore um programa cliente,
    desta vez, que utilize o protocolo \texttt{TCP}, e seja capaz de se comunicar
    com um servidor \texttt{TCP} remoto com endereço \texttt{68.183.28.251}. Este
    servidor executa uma segunda instância do processo \textit{echoserver}, desta
    vez utilizando, também, o protocolo \texttt{TCP}, na porta 6667. O aplicativo
    cliente deve ser capaz de:

    \begin{itemize}
      \item[1.] Enviar mensagens (com qualquer conteúdo) a cada 400 ms para o servidor
        pelo período de 20 s.
      \item[3.] Medir o intervalo de tempo ($ \Delta_t $) entre o momento em que
        o cliente faz a requisição ($ t_{req} $) e o momento em que recebe uma
        resposta ($ t_{res} $) do servidor ($ \Delta_t = t_{res} - t_{req} $).
      \item[4.] Salvar, em um arquivo texto simples (\texttt{.dat}), os valores de
        $ \Delta_t $ e $ t $ obtidos no item anterior.
    \end{itemize}

  \item[3)] De posse dos dados gerados pelos clientes dos dois aplicativos desenvolvidos
    nos exercícios anteriores:

    \begin{itemize}
      \item[1.] Elabore um gráfico de $ \Delta_t $ em função de $ t $ para a
        comunicação entre cliente e servidor para cada um dos casos, ou seja, uma
        curva para os resultados obtidos utilizando o protocolo TCP e outra para
        os resultados obtidos utilizando o protocolo UDP.
      \begin{itemize}
        \item[a.] Coloque as diferentes curvas no mesmo gráfico para que seja possível
          visualizar melhor as diferenças temporais entre as diferentes situações.
      \end{itemize}
      \item[2.] Calcule a média, variância, desvio padrão e os valores máximo e mínimo
        de $ \Delta_{t_{TCP}} $ e $ \Delta_{t_{UDP}} $ (dados temporais da comunicação
        TCP e UDP, respectivamente).
      \item[3.] Comente sobre as diferenças encontradas e suas possíveis causas.
    \end{itemize}
\end{itemize}

\section*{Relatório}

Os exercícios deste laboratório podem ser realizados em grupos de até \ebf{rdy}{dois}
alunos. Cada grupo deve elaborar um relatório detalhando cada atividade realizada,
incluindo comentários sobre as escolhas adotadas para a solução de cada exercício,
fluxogramas, esquemáticos e códigos de programação elaborados. Este relatório deve
ser entregue no formato de arquivo \ebt{rdy}{PDF}. Os arquivos de código-fonte também
devem ser entregues, de forma que seja possível reproduzir os resultados obtidos em
outros dispositivos.

\end{document}
